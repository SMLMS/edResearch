\section{Diskussion}
Mit Hilfe des ARCS Modells wurde überprüft, ob das entwickelte Lehrkonzept einen positiven Einfluss auf das Selbstbewusstsein der Studierenden hat und welchen Anteil bestimmte Lehrmethoden an dem Zuwachs des Selbstbewusstseins der Studierenden innerhalb eines Literaturseminars haben. Das Selbstbewusstsein der Studierenden konnte mit Hilfe des entwickelten Lehrkonzepts bei insgesamt 73\% der Studierenden gesteigert werden. Unter den Studierenden, die ihr Selbstbewusstsein als gering einstuften, zeigten sogar 80\% ein gesteigertes Selbstbewusstsein an. Als Hauptgrund für ein vermindertes Selbstbewusstsein gaben die meisten Studierenden eine unzureichende Vorbereitung auf den Stoff an. Nach Angaben der Studierenden halfen vier der zusätzlich angebotenen Lehrmethoden maßgeblich bei der Vorbereitung auf den Unterricht. Lediglich die Besetzung der Rolle des Diskussionsleiters durch Studierende hatte keinen Effekt auf die Vorbereitung. Durch die Methode der bedingten Wahrscheinlichkeiten wurde jedoch gezeigt, dass nur 72\% der Studierenden, denen die zusätzlichen Lehrmethoden bei der Vorbereitung geholfen haben, auch tatsächlich einen Zuwachs an Selbstbewusstsein bemerkten. Angst vor dem Dozenten gaben die wenigsten Studierenden als Ursache für ein vermindertes Selbstbewusstsein an. Allerdings konnte mit der Methode der bedingten Wahrscheinlichkeiten gezeigt werden, dass es bei >83\% der Studierenden, denen die Angst vor dem Dozenten genommen werden konnte, zu einem Zuwachs an Selbstbewusstsein kommt. Ein Abbau der Angst vor dem Dozenten erhöht das Selbstbewusstsein der Studierenden also effektiver, als eine Hilfe bei der Unterrichtsvorbereitung. Die zusätzlich eingeführten Lehrmethoden halfen allerdings weniger als der Hälfte der Studierenden, die Angst vor dem Dozenten abzubauen. Den größten Anteil hatte das praktische Arbeiten im Labor. Es half bei >36\% der Studierenden, die Angst vor dem Dozenten abzubauen. Dies lässt sich vermutlich durch die enge Zusammenarbeit zwischen Dozent und Studierenden erklären. Während der praktischen Übungen hatten die Studierenden über einen längeren Zeitraum einen direkten Kontakt zum Dozenten. Basierend auf dieser Annahme sollten zusätzlich Methoden eingeführt werden, die den Studierenden einen direkten Kontakt zum Dozenten ermöglichen, wie zum Beispiel Einzelgespräche vor den Präsentationen. Diese könnten dazu beitragen, dass die Angst vor dem Dozenten weiter abgebaut wird und so zu einem verstärkten Zuwachs an Selbstbewusstsein führen. Diese Annahme bedarf es allerdings noch zu überprüfen.