\section{Ergebnisse}

\subsection{Steigerung des Selbstbewusstseins durch aktivierende Lehrmethoden}
Zur Überprüfung der These, dass das entwickelte Lehrkonzept das Selbstbewusstsein der Studierenden in einem Literaturseminar steigern konnten, schätzten die Studierenden ihr Selbstbewusstsein in vorangegangenen Literaturseminaren ein. Zusätzlich gaben sie an, ob ihr Selbstbewusstsein im aktuellen Seminar gestiegen ist im Vergleich zu früheren Seminaren. Unter den befragten Studierenden gaben 45\% an, bisher ein geringes Selbstbewusstsein gehabt zu haben. Die zusätzlichen Lehrmethoden zeigten bei insgesamt 73\% der Studierenden einen positiven Einfluss auf das Selbstbewusstsein. Innerhalb der Gruppe der Studierenden, welche ihr Selbstbewusstsein als gering eingestuft hatten, berichteten 80\% der Studierenden von einem positiven Effekt der zusätzlichen Lehrmethoden auf ihr Selbstbewusstsein (siehe Abbildung \ref{fig:P_H_1-8_D_1-1}).

\begin{figure}[h!]
\begin{center}
	\includegraphics[trim=0cm 0cm 0cm 0cm,clip, page=1,width=0.8\linewidth]{../figures/P_H_1-8_D_1-1.pdf}
\end{center}
	\caption{ 80\% der Studierenden mit einem geringen Selbstbewusstsein berichten von einer Zunahme ihres Selbstbewusstseins bedingt durch die zusätzlichen Lehrmethoden.}
	\label{fig:P_H_1-8_D_1-1}
\end{figure}

\subsection{Ursachen eines geringen Selbstbewusstseins}
Als Ursache für ein vermindertes Selbstbewusstsein der Studierenden wurden vier Hypothesen (1.2.1 bis 1.2.4 im Anhang) aufgestellt, welche die Studierenden mit trifft zu (TRUE) oder trifft nicht zu (FALSE) charakterisieren sollten (siehe Abbildung \ref{fig:P_H_1-2}). Das Ergebnis umfasst die Aussagen aller Studierenden. Demnach leidet das Selbstbewusstsein der Studierenden am meisten durch unzureichende Vorbereitung (Hypothese 1.2.2 = TRUE: 67\%) und fehlendes Vorwissen (Hypothese 1.2.1 = TURE: 45\%). Angst, sich vor den Kommilitonen zu blamieren (Hypothese 1.2.3), gaben nur 36\% der Studierenden an. Angst, vor dem Dozenten etwas Falsches zu sagen (Hypothese 1.2.4), war bei 27\% der Studierenden ein Grund für ein vermindertes Selbstbewusstsein.   

\begin{figure}[h!]
\begin{center}
	\includegraphics[trim=0cm 0cm 0cm 0cm,clip, page=1,width=0.8\linewidth]{../figures/P_H_1-2.pdf}
\end{center}
	\caption{Gründe für ein vermindertes Selbstbewusstsein aller Studierenden in einem Literaturseminar.}
	\label{fig:P_H_1-2}
\end{figure}

\subsection{Die eingesetzten Lerhmethoden wirken sich positiv auf das Selbstbewusstsein der Studierenden aus}
Die Studierenden sollten einschätzen, welche der  zusätzlichen Lehrmethoden (Wiederholung in Form des {\it inverted class room}, Vorbereitung mit Hilfe der Referenten Zusammenfassung, praktische Übungen im Labor, Arbeit in Kleinstgruppen, Leitung der Diskussion durch einen Studierenden) einen positiven Einfluss auf ihr Selbstbewusstsein hatten, indem mindestens einer der zuvor aufgeführten Gründe für ein vermindertes Selbstbewusstsein an Einfluss verloren hat. Die Leitung der Diskussionsrunde durch Studierende führte bei keinem der Befragten zu einer Verringerung der Ursachen für ein vermindertes Selbstbewusstsein. Die restlichen Methoden des Lehrkonzepts hatten allesamt bei >90\% der Studierenden einen positiven Effekt auf die Ursachen für ein vermindertes Selbstbewusstsein (siehe Tabellen \ref{tab:3} bis \ref{tab:7} im Kapitel \nameref{chap:dataset}).

\subsection{Eine Reduktion der Ursachen für ein vermindertes Selbstbewusstsein bedingt effektiv die Zunahme des Selbstbewusstseins}\label{chap:effektiveZunahme}
Sobald der Einfluss einer der oben genannten Ursachen reduziert wird, steigert sich bei mindestens 73\% der Studierenden das Selbstbewusstsein (siehe Abbildung \ref{fig:P_H_1-8_D_reason}). Die Zunahme des Selbstbewusstseins (P(H=TRUE)) wird am effektivsten durch eine verminderte Angst vor dem Dozenten bedingt. So berichteten 83\% der Studierenden, denen die Angst vor dem Dozenten durch die zusätzlichen Lehrmethoden genommen worden ist, von einem gesteigerten Selbstbewusstsein. Eine Verminderung der Angst vor den Kommilitonen führt bei 75\% der Studierenden zu einem gesteigerten Selbstbewusstsein. Ein besseres Grundwissen sowie eine optimierte Vorbereitung führte jeweils bei 73\% der Studierenden zu einem gesteigerten Selbstbewusstsein.
\begin{figure}[h!]
\begin{center}
	\includegraphics[trim=0cm 0cm 0cm 0cm,clip, page=1,width=0.8\linewidth]{../figures/P_H_1-8_D_reason.pdf}
\end{center}
	\caption{Die Reduktion einer der Ursachen für ein vermindertes Selbstbewusstsein (P(D=TRUE)) bedingt effiktiv die Zunahme des Selbstbewusstseins (P(H=TRUE))}
	\label{fig:P_H_1-8_D_reason}
\end{figure}

\subsection{Das Lehrkonzept hilft den Studierenden bei der Vorbereitung des Unterrichts}
Eine unzureichende Vorbereitung auf den Stoff war für 67\% der Studierenden eine Ursache für ein vermindertes Selbstbewusstsein (siehe Abbildung \ref{fig:P_H_1-2}). Mit Ausnahme der Besetzung des Diskussionsleiters durch Studierende trug jede der zusätzlichen Lehrmethoden mit >81\% Wahrscheinlichkeit zu einer besseren Vorbereitung des Stoffes bei (siehe Abbildung \ref{fig:P_H_1-x-2}).
\begin{figure}[h!]
\begin{center}
	\includegraphics[trim=0cm 0cm 0cm 0cm,clip, page=1,width=0.8\linewidth]{../figures/P_H_1-x-2.pdf}
\end{center}
	\caption{Lehrmethoden, welche den Studierenden bei der Vorbereitung helfen.}
	\label{fig:P_H_1-x-2}
\end{figure}

\subsection{Die Angst vor dem Dozenten wird effektiv durch praktische Übungen in kleinen Gruppen in enger Zusammenarbeit mit dem Dozenten reduziert}
Es wurde gezeigt, dass bei Studierenden, denen die Angst vor dem Dozenten genommen worden ist, die Wahrscheinlichkeit für eine Zunahme des Selbstbewusstseins am größten ist (siehe Kapitel \ref{chap:effektiveZunahme}). Daher wird untersucht, welche der aktivierenden Lehrmethoden den Studierenden die Angst vor dem Dozenten nehmen.
\begin{figure}[h!]
\begin{center}
	\includegraphics[trim=0cm 0cm 0cm 0cm,clip, page=1,width=0.8\linewidth]{../figures/P_H_1-x-4.pdf}
\end{center}
	\caption{Lehrmethoden, welche den Studierenden die Angst vor dem Dozenten nehmen.}
	\label{fig:P_H_1-x-4}
\end{figure}
Abbildung \ref{fig:P_H_1-x-4} zeigt, dass die Studierenden die Angst vor dem Dozenten vor allem im Praktikum verloren haben. 35\% der Studierenden gaben an, dass das praktische Bearbeiten von Problemen unter der Anleitung des Dozenten zu einem Abbau der Angst vor dem Dozenten beitrug. Die restlichen Methoden führten jeweils bei 27\% der Studierenden zu einer verminderten Angst vor dem Dozenten. Die Besetzung des Diskussionsleiters durch Studierende führte bei keinem der Teilneher zu einer Reduktion der Angst vor dem Dozenten.