\section{Einleitung}
Literaturseminare mit anschließender Diskussion, in denen naturwissenschaftliche Veröffentlichungen vorgestellt werden, stellen für Studierende eine besondere Art der Prüfung dar. Sie können als offene Prüfung gesehen werden. Während sich der Referent eingehend mit dem Thema der vorzustellenden Literatur beschäftigt hat, ist das Auditorium meist weniger gut vorbereitet. Falsche Antworten, ein Missverständnis der naturwissenschaftlichen Zusammenhänge oder eine
Fehleinschätzung der Signifikanz der vorgestellten Arbeit werden direkt von den Kommilitonen wahrgenommen. Diese Situation kann zu einem inneren Konflikt führen, sich einerseits an der Diskussion zu beteiligen und eine falsche Antwort zu riskieren oder gegebenenfalls Wissenslücken zu offenbaren, andererseits sich der Diskussion zu verwehren, um sich nicht bloßzustellen. Dieser Konflikt wirkt sich direkt auf die Motivation der Studierenden aus, sich an der Diskussion der vorgestellten Arbeit zu beteiligen.\\
\noindent
Die vorliegende Arbeit verwendet das ARCS Modell zur Motivationsoptimierung im Rahmen eines Literaturseminars. Das Modell wurde von Keller und Kopp \cite{M.Keller.1987} entwickelt und erfolgreich in Lehrveranstaltungen mit physikalischem Inhalt angewendet \cite{AsksoyG.2016}. Es basiert auf dem Prinzip des Instruktionsdesigns, welches maßgeblich von Gabi Reinmann geprägt wurde \cite{reinmann2011instructional}. Das Akronym steht hierbei für vier motivationale Bedingungen: Aufmerksamkeit (attention), Relevanz (relevance), Erfolgszuversicht (confidence), Zufriedenheit (satisfaction). Das ARCS Modell wurde entwickelt, um Motivationsprobleme analytisch zu erfassen und zu lösen \cite{Keller.1987}. Hierzu führt Keller das Motivationsdesignmodell zur systematischen Entwicklung von Strategien zur Identifikation und Behebung von Motivationsproblemen ein. Das Motivationsmodell basiert auf vier Stufen: Der Definition eines spezifischen Motivationsproblems, dem Design einer potentiellen Strategie dem definierten Motivationsproblem entgegenzuwirken, der Entwicklung von Unterrichtseinheiten sowie Unterrichtsmaterialien, welche die Strategie zur Motivationssteigerung umsetzen und der Evaluation der entwickelten Strategie.\\
\noindent
Nach Kellers Definition sinkt die Erfolgszuversicht der Studierenden mit mangelndem Selbstbewusstsein, was sich nach dem ARCS Modell direkt negativ auf die Motivation auswirkt. Um Studierende mit einem geringeren Selbstbewusstsein zu einer erhöhten Teilname an der Diskussion zu motivieren, wurde im Rahmen der vorliegenden Studie ein Lehrkonzept entwickelt welches das Selbstbewusstsein der Studierenden innerhalb der Gruppe steigern soll. Das vorgestellte  Konzept basiert auf vier Säulen:
\begin{enumerate}
\item Festigung des Basiswissens durch Wiederholung
\item Hilfestellung bei der Unterrichtsvorbereitung durch Zusammenfassungen
\item Praktische Anwendung des Gelernten
\item Erhöhtes Gruppengefühl durch intensive Interaktion der Studierenden untereinander und mit dem Dozenten
\end{enumerate}
Zur Umsetzung der Motivationsstrategie des ARCS wurden neben den Seminarvorträgen aktivierende Lehrmethoden ({\it Kugellager} \cite{Leisen.2003}, {\it Mind-Mapping}, {\it inverted class room} \cite{Uzunboylu.2015, AsksoyG.2016}, Praktisches Arbeiten) eingesetzt. Diese dienten vor allem der Festigung des bisher Gelernten, sowie dem Abbau von Distanz zwischen den Studierenden untereinander und zum Dozenten. Zur Unterrichtsvorbereitung wurde von den vortragenden Studierenden zusätzlich zu ihrer Präsentation eine Zusammenfassung der vorgestellten Arbeit verfasst. Diese wurde im Vorfeld an die Präsentation ausgeteilt und von den übrigen Studierenden durchgearbeitet. Die Leitung des Seminars und der darauffolgenden Diskussion wurde von Studierenden übernommen, um die Angst vor dem Feedback des Dozenten zu verringern.\\
\noindent
Die Evaluation der entwickelten Strategie geschah mit Hilfe eines Fragenkatalogs zur Selbsteinschätzung des eigenen Selbstbewusstseins.