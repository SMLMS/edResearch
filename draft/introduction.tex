\section{Einleitung}
Literaturseminare in denen naturwissenschaftliche Veröffentlichungen vorgestellt werden mit anschließender Diskussion stellen für Studierende eine besondere Art der Prüfung dar. Sie können als offene Prüfung gesehen werden. Während sich der Referent eingehend mit dem Thema der vorzustellenden Literatur beschäftigt hat, ist das Auditorium meist weniger gut vorbereitet. Falsche Antworten, ein Missverständnis der naturwissenschaftlichen Zusammenhänge oder eine
Fehleinschätzung der Signifikanz der vorgestellten Arbeit werden direkt von den Kommilitonen wahrgenommen. Diese Situation kann zu einem inneren Konflikt führen, sich einerseits an der Diskussion zu beteiligen und eine falsche Antwort zu riskieren oder gegebenenfalls Wissenslücken zu offenbaren, andererseits sich der Diskussion zu verwehren, um sich nicht bloßzustellen. Dieser Konflikt wirkt sich direkt auf die Motivation der Studierenden aus, sich an der Diskussion der vorgestellten Arbeit zu beteiligen.\\
\noindent
Die vorliegende Arbeit verwendet das ARCS Modell zur Motivationsoptimierung im Rahmen eines Literaturseminars. Das Modell wurde von Keller und Kopp \cite{Kopp.1987} entwickelt. Es basiert auf dem Prinzip des Instruktionsdesigns welcher maßgeblich von Gabi geprägt wurde. Das Akronym steht hierbei für vier motivationale Bedingungen: Aufmerksamkeit (attention), Relevanz (relevance), Erfolgszuversicht (confidence), Zufriedenheit (satisfaction). Das ARCS Modell wurde entwickelt, um bestimmte Motivationsprobleme analytisch zu erfassen und zu Lösen \cite{Keller.1987}. Hierzu führt Keller das Motivationsdesignmodell zur systematischen Entwicklung von Strategien zur Identifikation und Behebung von Motivationsproblemen ein. Das Motivationsmodell basiert auf vier Stufen: Der Definition eines spezifischen Motivationsproblems, dem Design einer potentiellen Strategie dem definierten Motivationsproblem entgegenzuwirken, der Entwicklung von Unterrichtseinheiten sowie Unterrichtsmaterialien, welche die Strategie zur Motivationssteigerung umsetzen und der Evaluation der entwickelten Strategie.\\
\noindent
Nach Kellers Definition sinkt die Erfolgszuversicht der Studierenden mit Mangelndem Selbstbewusstsein, was sich nach dem ARCS Modell direkt negativ auf die Motivation auswirkt. Um Studierenden mit einem geringeren Selbstbewusstsein zu einer erhöhten Teilname an der Diskussion zu motivieren, wurde ein aktivierendes Lehrkonzept entwickelt welches das Selbstbewusstsein der Studierenden innerhalb der Gruppe steigern soll. Das Konzept basiert auf vier Säulen:
\begin{enumerate}
\item Festigung des Basiswissens durch Wiederholung
\item Ergebnisprotokollierung durch regelmäßige Zusammenfassungen
\item Praktische Anwendung des gelernten
\item Erhöhte Interaktion der Studierenden untereinander und mit dem Dozenten
\end{enumerate}
Zur Umsetzung der Motivationsstrategie wurden neben den Seminarvorträgen aktivierende Lehrmethoden (Kugellager \cite{Leisen.2003}, Mind-Mapping, inverted class room \cite{Uzunboylu.2015, AsksoyG.2016}, Praktisches Arbeiten) eingesetzt. Diese dienten vor allem der Wiederholung und Anwendung des Gelernten auf eine gegebene Fragestellung, sowie dem Abbau von Distanz zwischen den Studierenden untereinander und zum Dozenten.\\
\noindent
Die Evaluation der entwickelten Strategie geschah mit Hilfe eines Fragenkatalogs zur Selbsteinschätzung des eigenen Selbstbewusstseins.  

