\section{Methoden}
Die Studie erfasst das Selbstbewusstseins der Teilnehmer, sowie den Einfluss aktivierender Lehrmethoden auf das Selbstbewusstsein. Gemessen wird die subjektive Selbsteinschätzung an Hand einer Liste von Hypothesen, welche binär (trifft zu, trifft nicht zu) beantwortet werden können (siehe Kapitel \ref{chap:questionnaire}.

\subsection{Teilnehmer}
Insgesamt nahmen 11 Studierende an der Studie teil. Die Teilnehmer setzten sich zusammen aus Studierenden der Masterstudiengänge Chemie, Biochemie und Biophysik. Studiengang und Geschlecht der Teilnehmer (w/m/d) wurde im Rahmen der Studie nicht berücksichtigt.

\subsection{Studiendesign}
Der Kurs wurde bisher als reines Literaturseminar gehalten. Die Studierenden bekamen zu Beginn des Kurses einen Forschungsbericht zugeteilt. Der Bericht wurde von den Studierenden in Eigenarbeit gelesen. Die Studierenden bereiteten eine 20-minütige Präsentation des Forschungsberichts vor, in welcher der Bericht einem Publikum bestehend aus den übrigen Studierenden und dem Dozenten vorgestellt wurde. Im Anschluss an die Präsentation wurde die Arbeit im Plenum diskutiert.\\
\noindent
Der diesjährige Kurs wurde um folgende Methoden erweitert: Zu Beginn des Kurses wurde ein Brainstorming der Studierenden zum Thema des Literaturseminars in Form eines {\it Kugellagers} durchgeführt. Der Austausch fand ohne weitere Vorbereitung der Studierenden in Zweier-Gruppen statt. Nach jeweils 2 Minuten wurden die Kleinstgruppen neu zusammengestellt. Das kollektive Vorwissen wurde in Form einer {\it Mind-Map} an der Tafel festgehalten. Für die folgende Unterrichtseinheit wiederholten die Studierenden ihr Vorwissen aus der vorangegangenen Vorlesung {\it Einführung in die Einzelmolekülspektroskopie und hochauflösende Mikroskopie} in Heimarbeit. Als Lernhilfe dienten hierbei die bekannten Unterrichtsmaterialien aus der Vorlesung. Dieses Vorwissen wurde in der nächsten Unterrichtseinheit in Form eines {\it inverted class room} gemeinsam wiederholt und gefestigt.\\
\noindent
Während der eigentlichen Seminarphase der Veranstaltung bereiteten die Studierenden zusätzlich zu den Präsentationen eine schriftliche Zusammenfassung der Forschungsarbeit vor. Diese wurde dem Publikum vor der Präsentation ausgeteilt und diente sowohl der Vorbereitung, als auch der Ergebnissicherung.\\
\noindent
Die Rolle des Diskussionsleiters wurde nicht vom Dozenten sondern den Studierenden übernommen. Der Diskussionsleiter stellt zu Beginn des Seminars den Referenten vor und leitet im Anschluss an die Präsentation die Diskussion. Kurs-begleitend wurde ein Praktikum im Forschungslabor durchgeführt, in welchem die Teilnehmer in kleinen Gruppen (ca. 3-4 Personen) das Erlernte unter Zusammenarbeit mit dem Dozenten auf eine praktische Fragestellung übertragen konnten.

\subsection{Studiendurchführung}
Als Messinstrument dient eine Sammlung an Thesen (siehe \nameref{chap:questionnaire}), deren Wahrheitsgehalt die Studierenden in einem geschlossen, binären multiple choice (FALSE = trifft nicht zu; TRUE = trifft zu) bewerteten. Die Thesensammlung wurde am Ende der Lehrveranstaltung anonym von den Teilnehmern ausgefüllt. Ungültige oder nicht auswertbare Antworte sind als NAN ({it not a number}) deklariert.\\
\noindent
Die Studierenden definierten anonym ihr Selbstbewusstsein. Hierbei ist die Definition des Selbstbewusstseins in vorangegangenen reinen Literaturseminaren die unabhängige Variable, an Hand derer die Studierenden in eine Gruppe mit starkem Selbstbewusstsein und eine mit schwachem Selbstbewusstsein unterteilt werden. Weiter definierten die Studierenden, ob sich ihr Selbstbewusstsein durch die zusätzlichen Lehrmethoden gesteigert hat. Die abhängige Variable der Selbstbewusstseinssteigerung wird statistisch analysiert, indem die Fraktion an Studierenden ermittelt wird, bei denen aktivierende Lehrmethoden zu einer Steigerung des Selbstbewusstseins beitrugen. Hierbei werden die Studierenden, bedingt durch ihr anfängliches Selbstbewusstsein, in zwei Gruppen unterteilt.\\
\noindent
In einem zweiten Teil sollen die Lehrmethoden identifiziert werden, die zur Steigerung des Selbstbewusstseins geführt haben. Die unabhängige Variable ist hierbei eine positive Selbstbewusstseinssteigerung im vorangegangenem Test. Hierzu benannten die Studierenden, bei denen eine Selbstbewusstseinssteigerung eintrat, die verursachende Lehrmethode als abhängige Variable. Hierbei waren Mehrfachnennungen möglich. Im zweiten Teil werden die Antworten aller Teilnehmer berücksichtigt, die einen Zuwachs an Selbstbewusstsein dokumentierten, unabhängig von der Einteilung im ersten Teil der Studie.

\subsection{Datenauswertung}
Zur Datenauswertung wurden die Frequenzen positiver Antworten (trifft zu) ermittelt. Die Überprüfung der Hypothesen geschah über die Methode der {\it Bedingten Wahrscheinlichkeit} \cite{Eddy.2004}. Im Folgenden werden die verwendeten Notationen vorgestellt.\\
\noindent
Die Wahrscheinlichkeit $P$, dass eine Hypothese $H$ zutrifft, wird notiert als
\begin{align}
	\label{eq:Prob1}
	P(H)
\end{align}

\noindent
Die Wahrscheinlichkeit $P$, dass eine Hypothese $D$ zutrifft, wird notiert als
\begin{align}
\label{eq:Prob2}
	P(D)
\end{align}

\noindent
Für den Fall, dass die Wahrscheinlichkeit $P(H)$ von der Bedingung abhängt, dass zuvor Hypothese $D$ zutreffend war, wird die bedingte Wahrscheinlichkeit für ein Zutreffen von $H$ eingeführt:
\begin{align}
\label{eq:ConProb}
	P(H|D)
\end{align}

\noindent
Sind die Wahrscheinlichkeiten aus Gleichungen \eqref{eq:Prob1},\eqref{eq:Prob2} und \eqref{eq:ConProb} bekannt, so kann über den Satz von Bayes die bedingte Wahrscheinlichkeit $P(D|H)$ berechnet werden \cite{Eddy.2004}:
\begin{align}
\label{eq:Bayes}
	P(D|H) = \frac{P(D)P(H|D)}{P(H)}
\end{align}

\subsection{Verfügbarkeit der Studie}
Der anonyme Datensatz, sowie die entwickelte Auswerteroutine stehen frei zur Verfügung \url{https://github.com/SMLMS/edResearch}.