\section*{Fragebogen}\label{chap:questionnaire}
Die Studierenden sollten folgende Hypothesen bewerten. Zur Bewertung diente eine binäre Antwortmöglichkeit: trifft nicht zu (FALSE) und trifft zu (TRUE). Unvollständige Antworten und Antworten, die nicht auszuwerten waren, wurden mit NAN ({\it not a number}) bewertet.

\subsection*{Selbsteinschätzung des eigenen Selbstbewusstseins}
Bitte bewerten Sie folgende Aussage:
\subsubsection*{Hypothese 1.1}
\label{chap:H_1-1}
Bei einer Diskussion wissenschaftlicher Fragestellungen innerhalb einer Gruppe vor dem Dozenten fühle ich mich unwohl und halte mich eher zurück, um nichts Falsches zu sagen.

\subsection*{Ursachen für ein vermindertes Selbstbewusstsein innerhalb einer Gruppe}
Sollte die Aussage aus \nameref{chap:H_1-1} auf Sie zutreffen, benennen Sie die Ursache für Ihre Antwort auf \nameref{chap:H_1-1}. Mehrfachnennungen sind möglich.
\subsubsection*{Hypothese 1.2.1}
Wenn ich ein Problem nicht komplett verstehe, weil mir zum Beispiel grundlegendes Basiswissen fehlt, werde ich ängstlich.
\subsubsection*{Hypothese 1.2.2}
Es beunruhigt mich, eine schwierige Aufgabe zu bearbeiten, wenn ich nicht sicher bin, dass ich ausreichend vorbereitet bin.
\subsubsection*{Hypothese 1.2.3}
Arbeiten, bei denen meine Fähigkeiten vor einem großen Publikum mit meist unbekannten Teilnehmern auf die Probe gestellt werden, mag ich nicht.
\subsubsection*{Hypothese 1.2.4}
Bei einer direkten Diskussion mit dem Dozenten vor dem Auditorium bin in verunsichert, weil ich Angst habe etwas Falsches zu sagen.


\subsection*{Festigung des Basiswissens durch Wiederholung}
Bitte bewerten Sie die folgende Hypothese:\\
\noindent
Die Wiederholung des Basiswissens zu Beginn des Seminars in Form eines {\it inverted class rooms} ...
\subsubsection*{Hypothese 1.3.1}
hat mir bei dem Verständnis der diskutierten Daten geholfen.
\subsubsection*{Hypothese 1.3.2}
hat mir bei der Bewertung der wissenschaftlichen Arbeit, welche im Seminar vorgestellt wurde, geholfen.
\subsubsection*{Hypothese 1.3.3}
hat mir die Angst genommen, vor der Gruppe etwas Falsches zu sagen.
\subsubsection*{Hypothese 1.3.4}
hat mir die Angst genommen, vor der Gruppe durch den Dozenten bloßgestellt zu werden.


\subsection*{Arbeiten mit Referenten-Abstract}
Bitte bewerten Sie die folgende Hypothese:\\
\noindent
Die Literatur-Arbeit mit der Referenten-Zusammenfassung des Artikels vorangehend an das Seminar ...
\subsubsection*{Hypothese 1.4.1}
hat mir bei dem Verständnis der diskutierten Daten geholfen.
\subsubsection*{Hypothese 1.4.2}
hat mir bei der Bewertung der wissenschaftlichen Arbeit, welche im Seminar vorgestellt wurde, geholfen.
\subsubsection*{Hypothese 1.4.3}
hat mir die Angst genommen, vor der Gruppe etwas Falsches zu sagen.
\subsubsection*{Hypothese 1.4.4}
hat mir die Angst genommen, vor der Gruppe durch den Dozenten bloßgestellt zu werden.


\subsection*{Praktika begleitend zu Literaturseminaren}
Bitte bewerten Sie die folgende Hypothese:\\
\noindent
Der praktische Umgang mit der Thematik im Labor parallel zum Literaturseminar ...
\subsubsection*{Hypothese 1.5.1}
hat mir bei dem Verständnis der diskutierten Daten geholfen.
\subsubsection*{Hypothese 1.5.2}
hat mir bei der Bewertung der wissenschaftlichen Arbeit, welche im Seminar vorgestellt wurde, geholfen.
\subsubsection*{Hypothese 1.5.3}
hat mir die Angst genommen, vor der Gruppe etwas Falsches zu sagen.
\subsubsection*{Hypothese 1.5.4}
hat mir die Angst genommen, vor der Gruppe durch den Dozenten bloßgestellt zu werden.


\subsection*{Gruppenarbeit in Literaturseminaren}
Bitte bewerten Sie die folgende Hypothese:\\
\noindent
Die Laborarbeit in Kleinstgruppen ...
\subsubsection*{Hypothese 1.6.1}
hat mir bei dem Verständnis der diskutierten Daten geholfen.
\subsubsection*{Hypothese 1.6.2}
hat mir bei der Bewertung der wissenschaftlichen Arbeit, welche im Seminar vorgestellt wurde, geholfen.
\subsubsection*{Hypothese 1.6.3}
hat mir die Angst genommen, vor der Gruppe etwas Falsches zu sagen.
\subsubsection*{Hypothese 1.6.4}
hat mir die Angst genommen, vor der Gruppe durch den Dozenten bloßgestellt zu werden.


\subsection*{Leitung der Diskussion durch einen Chair}
Bitte bewerten Sie die folgende Hypothese:\\
\noindent
Die Besetzung der Rolle des Diskussionsleiters durch einen Studierenden ...
\subsubsection*{Hypothese 1.7.1}
hat mir bei dem Verständnis der diskutierten Daten geholfen.
\subsubsection*{Hypothese 1.7.2}
hat mir bei der Bewertung der wissenschaftlichen Arbeit, welche im Seminar vorgestellt wurde, geholfen.
\subsubsection*{Hypothese 1.7.3}
hat mir die Angst genommen, vor der Gruppe etwas Falsches zu sagen.
\subsubsection*{Hypothese 1.7.4}
hat mir die Angst genommen, vor der Gruppe durch den Dozenten bloßgestellt zu werden.

\subsection*{Selbsteinschätzung des eigenen Selbstbewusstsein}
Die besuchte Veranstaltung weicht in ihrem Aufbau von einem klassischen Literaturseminar ab. Die behandelte Thematik sowie die Datenanalyse wurde innerhalb eines Praktikums im Labor in kleinen Gruppen bearbeitet. Im Vorfeld zu den Seminarvorträgen wurden die wichtigsten Ergebnisse der Arbeiten in einer einseitigen Zusammenfassung aufbereitet und ausgeteilt.
Die Diskussion wurde in ein kleines Rollenspiel integriert, in dem ein Studierender und nicht der Dozent die Rolle des Chairs übernimmt. Vor diesem Hintergrund bewerten Sie bitte die folgende Aussage:

\subsubsection*{Hypothese 1.8}
Der Aufbau der Veranstaltung hat dazu beigetragen, dass ich mich bei einer Diskussion wissenschaftlicher Fragestellungen innerhalb einer Gruppe vor dem Dozenten sicherer fühle?