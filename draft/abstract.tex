\begin{abstract}
Literaturseminare, bei denen komplexe naturwissenschaftliche Studien vorgestellt und diskutiert werden, stellen für Studierende eine Herausforderung dar. Sie erfordern ein fundiertes Grundwissen der Materie und eine ausgiebige Einarbeitung in das Themengebiet. Fehlinterpretationen im Verlauf der Diskussion werden nicht nur von den Kommilitonen, sondern auch vom Dozenten wahrgenommen.  Daher kann mangelndes Selbstbewusstsein als einer der Hauptgründe für eine geringe Lernmotivation und verminderte Beteiligung von Studierenden in Literaturseminaren angenommen werden. Im Rahmen der Studie wird, basierend auf dem von Keller und Kopp entwickelten ARCS Modell, ein  Seminarkonzept entwickelt, welches gezielt das Selbstbewusstsein der Studierenden stärkt. Das vorgestellte Konzept steigert das das Verständnis, unterstützt die studierenden in der Vorbereitung des Unterrichts und baut hilft die Distanz zwischen den Kommilitonen untereinander und zwischen den Studierenden und dem Dozenten abzubauen. Die Evaluation der Studie zeigt, dass das Konzept vor allem bei der Vorbereitung des Unterrichts erfolgreich ist. 
\end{abstract}